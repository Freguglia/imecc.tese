\usepackage{lmodern}
\usepackage[T1]{fontenc}
\usepackage[utf8]{inputenc}
\usepackage{lastpage}
\usepackage{indentfirst}
\usepackage{color,xcolor}
\usepackage{graphicx}
\usepackage{microtype}

\usepackage[subentrycounter,seeautonumberlist,nonumberlist=true]{glossaries}

%\usepackage[brazilian]{backref}
\usepackage[alf,
	    abnt-repeated-author-omit=yes,
	    abnt-etal-list=0]{abntex2cite}
\usepackage{lipsum}
% ----------------------------------------------------------
% ----------------------------------------------------------
% PACOTES PESSOAIS (USADOS PELO AUTOR -- acrescente aqui seus pacotes)
% ----------------------------------------------------------
\usepackage[portuguese,onelanguage]{algorithm2e}	% para inserir algoritmos (longend,vlined)
% \usepackage{amsbsy}			% para símbolos matemáticos em negrito
% \usepackage{amscd}			% para diagramas
% \usepackage{amsfonts}			% fontes AMS
% \usepackage{amsmath}			% facilidades matemáticas
% \usepackage{amssymb}			% para os símbolos mais antigos
% \usepackage{amstext}			% para fragmentos tipo texto em modo matemático
\usepackage{amsthm}			% para teoremas
\usepackage{hyperref}			% Amplo suporte para hipertexto em LaTeX
\usepackage{cleveref}			% Referência cruzada inteligente
\usepackage{dsfont}			% para o estilo de conjuntos de números $\mathds{R}$
% \usepackage{ifthen}			% comandos de condição em LaTeX
\usepackage{listings}           	% para inserir códigos de outras linguagens de programação
% \usepackage{lscape}             	% para imprimir alguma página no formato paisagem
\usepackage{mathabx}			% conjunto de simbolos matemáticos
% \usepackage{mathrsfs}			% suporte para fontes RSFS
\usepackage{pdfpages}
\usepackage{verbatim}
