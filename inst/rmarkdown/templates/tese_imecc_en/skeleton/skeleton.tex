\documentclass[
	oldfontcommands,
	sumario=abnt-6027-2012,
	12pt,
	openright,
	oneside,
	a4paper,
	english,
	english
	]{imecc-unicamp}

\usepackage{color}
\usepackage{fancyvrb}
\newcommand{\VerbBar}{|}
\newcommand{\VERB}{\Verb[commandchars=\\\{\}]}
\DefineVerbatimEnvironment{Highlighting}{Verbatim}{commandchars=\\\{\}}
% Add ',fontsize=\small' for more characters per line
\usepackage{framed}
\definecolor{shadecolor}{RGB}{248,248,248}
\newenvironment{Shaded}{\begin{snugshade}}{\end{snugshade}}
\newcommand{\AlertTok}[1]{\textcolor[rgb]{0.94,0.16,0.16}{#1}}
\newcommand{\AnnotationTok}[1]{\textcolor[rgb]{0.56,0.35,0.01}{\textbf{\textit{#1}}}}
\newcommand{\AttributeTok}[1]{\textcolor[rgb]{0.77,0.63,0.00}{#1}}
\newcommand{\BaseNTok}[1]{\textcolor[rgb]{0.00,0.00,0.81}{#1}}
\newcommand{\BuiltInTok}[1]{#1}
\newcommand{\CharTok}[1]{\textcolor[rgb]{0.31,0.60,0.02}{#1}}
\newcommand{\CommentTok}[1]{\textcolor[rgb]{0.56,0.35,0.01}{\textit{#1}}}
\newcommand{\CommentVarTok}[1]{\textcolor[rgb]{0.56,0.35,0.01}{\textbf{\textit{#1}}}}
\newcommand{\ConstantTok}[1]{\textcolor[rgb]{0.00,0.00,0.00}{#1}}
\newcommand{\ControlFlowTok}[1]{\textcolor[rgb]{0.13,0.29,0.53}{\textbf{#1}}}
\newcommand{\DataTypeTok}[1]{\textcolor[rgb]{0.13,0.29,0.53}{#1}}
\newcommand{\DecValTok}[1]{\textcolor[rgb]{0.00,0.00,0.81}{#1}}
\newcommand{\DocumentationTok}[1]{\textcolor[rgb]{0.56,0.35,0.01}{\textbf{\textit{#1}}}}
\newcommand{\ErrorTok}[1]{\textcolor[rgb]{0.64,0.00,0.00}{\textbf{#1}}}
\newcommand{\ExtensionTok}[1]{#1}
\newcommand{\FloatTok}[1]{\textcolor[rgb]{0.00,0.00,0.81}{#1}}
\newcommand{\FunctionTok}[1]{\textcolor[rgb]{0.00,0.00,0.00}{#1}}
\newcommand{\ImportTok}[1]{#1}
\newcommand{\InformationTok}[1]{\textcolor[rgb]{0.56,0.35,0.01}{\textbf{\textit{#1}}}}
\newcommand{\KeywordTok}[1]{\textcolor[rgb]{0.13,0.29,0.53}{\textbf{#1}}}
\newcommand{\NormalTok}[1]{#1}
\newcommand{\OperatorTok}[1]{\textcolor[rgb]{0.81,0.36,0.00}{\textbf{#1}}}
\newcommand{\OtherTok}[1]{\textcolor[rgb]{0.56,0.35,0.01}{#1}}
\newcommand{\PreprocessorTok}[1]{\textcolor[rgb]{0.56,0.35,0.01}{\textit{#1}}}
\newcommand{\RegionMarkerTok}[1]{#1}
\newcommand{\SpecialCharTok}[1]{\textcolor[rgb]{0.00,0.00,0.00}{#1}}
\newcommand{\SpecialStringTok}[1]{\textcolor[rgb]{0.31,0.60,0.02}{#1}}
\newcommand{\StringTok}[1]{\textcolor[rgb]{0.31,0.60,0.02}{#1}}
\newcommand{\VariableTok}[1]{\textcolor[rgb]{0.00,0.00,0.00}{#1}}
\newcommand{\VerbatimStringTok}[1]{\textcolor[rgb]{0.31,0.60,0.02}{#1}}
\newcommand{\WarningTok}[1]{\textcolor[rgb]{0.56,0.35,0.01}{\textbf{\textit{#1}}}}

% Essenciais
\usepackage{lmodern}
\usepackage[T1]{fontenc}
\usepackage[utf8]{inputenc}
\usepackage{lastpage}
\usepackage{indentfirst}
\usepackage{color,xcolor}
\usepackage{graphicx}
\usepackage{microtype}

% Úteis
\usepackage[onelanguage]{algorithm2e}
\usepackage{amsmath}
\usepackage{amsthm}
\usepackage{hyperref}
\usepackage{cleveref}
\usepackage{dsfont}
\usepackage{listings}
\usepackage{pdfpages}
\usepackage{verbatim}
\usepackage[subentrycounter,seeautonumberlist,nonumberlist=true]{glossaries}
\usepackage[alf,
	    abnt-repeated-author-omit=yes,
	    abnt-etal-list=0]{abntex2cite}
\usepackage{lipsum}

  \usepackage{amsfonts}
  \usepackage{amssymb}


\titulo{Título do seu Trabalho Acadêmico}
\curso{Estatística}

\setboolean{ABNTEXotherlanguage}{true}
\titulootherlang{Title of your Academic Work}
\cursootherlang{Statistics}

  \autor[autora]{Nome do Aluno}
      \titulacao{Mestra}
    \titulation{Master}
    \tipotrabalho{Dissertação}
    \typework{Dissertation}
  

  \orientador[Orientadora]{Nome do Orientador}


\data{2021}

% \theoremstyle{plain}
\newtheorem{theorem}{Teorema}%[chapter]
\newtheorem{lemma}{Lema}%[chapter]
\providecommand*{\lemmaautorefname}{Lema}
\newtheorem{proposition}{Proposição}%[chapter]
\providecommand*{\propositionautorefname}{Proposição}
\newtheorem{corollary}{Corolário}%[chapter]
\providecommand*{\corollaryautorefname}{Corolário}
\newtheorem{conjecture}{Conjectura}%[chapter]
\providecommand*{\conjectureautorefname}{Conjectura}
\newtheorem{definition}{Definição}%[chapter]
\providecommand*{\definitionautorefname}{Definição}
\newtheorem{notation}{Notação}%[chapter]
\providecommand*{\notationautorefname}{Notação}
\newtheorem{remark}{Observação}%[chapter]
\providecommand*{\remarkautorefname}{Observação}
\newtheorem{example}{Exemplo}%[chapter]
\providecommand*{\exampleautorefname}{Exemplo}
\newtheorem{note}{Nota}%[chapter]
\providecommand*{\noteautorefname}{Nota}

\lstset{
  language=C++,
  basicstyle=\ttfamily,
  keywordstyle=\color{blue},
  stringstyle=\color{verde},
  commentstyle=\color{red},
  extendedchars=true,
  showspaces=false,
  showstringspaces=false,
  numbers=left,
  numberstyle=\tiny,
  breaklines=true,
  backgroundcolor=\color{green!10},
  breakautoindent=true,
  fontadjust=false
}

\definecolor{blue}{RGB}{41,5,195}
\definecolor{verde}{rgb}{0,0.5,0}

\makeatletter
\hypersetup{
  pdftitle={\@title},
  pdfauthor={\@author},
  pdfsubject={%
    \imprimirtipotrabalho\ apresentada ao Instituto de Matemática, Estatística %
    e Computação Científica da Universidade Estadual de Campinas como parte dos %
    requisitos exigidos para a obtenção do título de \imprimirtitulacao\ em %
    \imprimircurso.
  },
  pdfcreator={LaTeX with unicamp-abnTeX2},
  pdfkeywords={abnt}{latex}{abntex}{abntex2}{trabalho acadêmico},
  colorlinks=true,
  linkcolor=blue,
  citecolor=blue,
  filecolor=magenta,
  urlcolor=blue,
  bookmarksdepth=4
}
\makeatother

\everymath{\displaystyle}

\renewcommand{\sin}{\mathrm{sen}}
\renewcommand{\tan}{\mathrm{tg}}
\renewcommand{\csc}{\mathrm{cossec}}
\renewcommand{\cot}{\mathrm{cotg}}

\DeclareMathOperator{\posto}{\mathrm{posto}}
\DeclareMathOperator{\conv}{\mathrm{conv}}
\DeclareMathOperator{\diag}{\mathrm{diag}}
\DeclareMathOperator{\argmin}{\mathrm{arg}\min}
\DeclareMathOperator{\argmax}{\mathrm{arg}\max}

\setlength{\parindent}{2.0cm}
\setlength{\parskip}{0.2cm}
\newsubfloat{figure}
\providecommand*{\subfigureautorefname}{Subfigura}

  \makeindex

  \makeglossaries
  \newglossaryentry{pai}{
      name={pai},
      plural={pai},
      description={este é uma entrada pai, que possui outras
      subentradas.}
  }
  \newglossaryentry{componente}{
      name={componente},
      plural={componentes},
      parent=pai,
      description={descriação da entrada componente.}
  }
  \newglossaryentry{filho}{
      name={filho},
      plural={filhos},
      parent=pai,
      description={isto é uma entrada filha da entrada de nome
  	\gls{pai}. Trata-se de uma entrada irmã da entrada
  	\gls{componente}.
      }
  }
  \newglossaryentry{equilibrio}{
      name={equilíbrio da configuração},
      see=[veja também]{componente},
      description={consistência entre os \glspl{componente}}
  }
  \newglossaryentry{latex}{
      name={LaTeX},
      description={ferramenta de computador para autoria de
      documentos criada por D. E. Knuth}
  }
  \newglossaryentry{abntex2}{
      name={abnTeX2},
      see=latex,
      description={suíte para LaTeX que atende os requisitos das
      normas da ABNT para elaboração de documentos técnicos e
      científicos brasileiros}
  }
  \renewcommand*{\glsseeformat}[3][\seename]{\textit{#1}
   \glsseelist{#2}}
  \renewcommand{\glossaryname}{Glossário}
  \providetranslation{Glossary}{Glossário}
  \providetranslation{Acronyms}{Siglas}
  \providetranslation{Notation (glossaries)}{Notação}
  \providetranslation{Description (glossaries)}{Descrição}
  \providetranslation{Symbol (glossaries)}{Símbolo}
  \providetranslation{Page List (glossaries)}{Lista de Páginas}
  \providetranslation{Symbols (glossaries)}{Símbolos}
  \providetranslation{Numbers (glossaries)}{Números}
  \setglossarystyle{index}

\begin{document}

\selectlanguage{english}
\frenchspacing

\pretextual
\imprimirprimeirafolha

  \imprimirfolhaderosto

\begin{fichacatalografica}
      \begin{center}
  {\ABNTEXchapterfont\large A ficha catalográfica será fornecida pela biblioteca}
    \end{center}
  \end{fichacatalografica}

\begin{folhadeaprovacao}
      \centering{\ABNTEXchapterfont\large A folha de aprovação será fornecida pela Secretaria de Pós-Graduação}
  \end{folhadeaprovacao}

  \begin{dedicatoria}
     \vspace*{\fill}
     \centering
     \noindent
     \textit{
        (Opcional) Escreva aqui sua dedicatória, Exclua essa variável do
        cabeçalho caso não queira incluir uma
     }
     \vspace*{\fill}
  \end{dedicatoria}

  \begin{agradecimentos}
  Inserir os agradecimentos, sem esquecer dos órgãos de fomento!

  This study was financed in part by the Coordenação de Aperfeiçoamento
  de Pessoal de Nível Superior - Brasil (CAPES) - Finance Code 001.

  Para bolsistas FAPESP: Incluir as seguintes informações. Processo nº
  aaaa/nnnnn-d, Fundação de Amparo à Pesquisa do Estado de São Paulo
  (FAPESP). As opiniões, hipóteses e conclusões ou recomendações
  expressas neste material são de responsabilidade do(s) autor(es) e não
  necessariamente refletem a visão da FAPESP.
  \end{agradecimentos}

  \begin{epigrafe}
      \vspace*{\fill}
      \begin{flushright}
  	\textit{(Opcional)\\
Quem sabe\\
faz ao vivo.\\
(Silva, Fausto)}
      \end{flushright}
  \end{epigrafe}
  
\setlength{\absparsep}{18pt}
\begin{resumo}[Resumo]
 \begin{otherlanguage*}{brazil}
    (Obrigarório) Escreva aqui o seu resumo em português. Escreva aqui o
    seu resumo em português. Escreva aqui o seu resumo em português.
    Escreva aqui o seu resumo em português. Escreva aqui o seu resumo em
    português. Escreva aqui o seu resumo em português.

    \textbf{Palavras-chave}: latex. abntex. editoração de texto.
 \end{otherlanguage*}
\end{resumo}
\begin{resumo}[Abstract]
 \begin{otherlanguage*}{english}
    (Obrigatório) Versão em inglês do resumo (Abstract). Versão em
    inglês do resumo (Abstract). Versão em inglês do resumo (Abstract).
    Versão em inglês do resumo (Abstract). Versão em inglês do resumo
    (Abstract). Versão em inglês do resumo (Abstract).

    \textbf{Keywords}: latex. abntex. text editoration.
 \end{otherlanguage*}
\end{resumo}

  \pdfbookmark[0]{\listfigurename}{lof}
  \listoffigures*
  \cleardoublepage

  \pdfbookmark[0]{\listtablename}{lot}
  \listoftables*
  \cleardoublepage

  \begin{siglas}
        \item[UNICAMP] Universidade Estadual de Campinas
        \item[IMECC] Instituto de Matemática
      \end{siglas}

\begin{simbolos}
    \item[\(\mathds{R}\)] Conjunto dos número reais
    \item[\(\frac{\partial f}{\partial x}\)] Derivada parcial de \(f\)
com respeito a \(x\)
  \end{simbolos}

\pdfbookmark[0]{\listalgorithmcfname}{loa}
\listofalgorithms
\cleardoublepage

\pdfbookmark[0]{\lstlistlistingname}{lol}
\begin{KeepFromToc}
\lstlistoflistings
\end{KeepFromToc}
\cleardoublepage

\pdfbookmark[0]{\contentsname}{toc}
\tableofcontents*
\cleardoublepage

\textual

\chapter*[Introdução]{Introdução}
\addcontentsline{toc}{chapter}{Introdução}

Esse é um template para para a escrita de teses no formato padrão do
IMECC - UNICAMP utilizando Rmarkdown. Utilize esse arquivo no formato
\texttt{.Rmd} como referência para o desenvolvimento do seu texto.

Ao longo desse arquivo são apresentados diversos exemplos de como
inserir e formatar vários tipos de elementos, como imagens, algoritmos,
tabelas, etc.

Apesar de ser baseado em um arquivo \texttt{Rmarkdown}, no processo de
construção do documento final, o arquivo é convertido para LaTeX de
maneira automática, o que faz com que seja possível utilizar toda e
qualquer funcionalidade do LaTeX com a mesma sintaxe sem nenhum
problema. Por exemplo, utlizando o ambiente de equações do LaTeX no
arquivo \texttt{.Rmd}

\begin{verbatim} 
\begin{equation}
  f(x) = \frac{1}{\sqrt{2\pi\sigma^2}} e^{-\frac{x^2}{2}}.
\end{equation}
\end{verbatim}

resulta na equação renderizada abaixo no documento final gerado.

\begin{equation}
  f(x) = \frac{1}{\sqrt{2\pi\sigma^2}} e^{-\frac{x^2}{2}}.
\end{equation} Portanto, esse modelo em \texttt{Rmarkdown} estende as
funcionalidades do LaTeX do modelo anterior, com a possibilidade de
utilizar chunks em \texttt{R} ou \texttt{Python} para produzir
informações no texto. Isso possibilita manter seu projeto mais
organizado com o código junto ao texto e garante a reproducibilidade do
trabalho.

\chapter{Utilizando chunks em R para gerar tabelas e gráficos}

A principal diferença entre escrever o trabalho em LaTeX ou
\texttt{Rmarkdown} é a possibilidade de escrever chunks com códigos que
são avaliados e os resultados são automaticamente incluídos no texto.

Na maioria das vezes, os resultados obtidos são acresecnetados ao texto
por meio de Tabelas ou de Figuras. Nessa seção temos exemplos desses
dois tipos de conteúdo.

\section{Figuras}

Podemos gerar gráficos diretamente no documento através de chunks em R.
Utilize as opções no chunk para controlar parâmetros como tamanho da
figura, legenda, etc. Caso não seja especificado, será utilizado o
parâmetro \texttt{echo\ =\ FALSE} (definido globalmente no primeiro
chunk de configuração) que não adiciona os códigos ao corpo do
documento, mas nesse exemplo está sendo usado \texttt{echo=TRUE} para
ilustração.

\begin{Shaded}
\begin{Highlighting}[]
\FunctionTok{library}\NormalTok{(ggplot2)}
\FunctionTok{ggplot}\NormalTok{(iris, }\FunctionTok{aes}\NormalTok{(}\AttributeTok{x =}\NormalTok{ Petal.Width,}
                 \AttributeTok{y =}\NormalTok{ Sepal.Width, }
                 \AttributeTok{color =}\NormalTok{ Species)) }\SpecialCharTok{+}
  \FunctionTok{geom\_point}\NormalTok{()}
\end{Highlighting}
\end{Shaded}

\begin{figure}[h]

{\centering \includegraphics{skeleton_files/figure-latex/exemplo_figura-1} 

}

\caption{Gráfico gerado com o pacote ggplot.}\label{fig:exemplo_figura}
\end{figure}

Para colocar múltiplos gráficos na mesma figura, basta utilizar a opção

\begin{Shaded}
\begin{Highlighting}[]
\FunctionTok{ggplot}\NormalTok{(iris, }\FunctionTok{aes}\NormalTok{(}\AttributeTok{x =}\NormalTok{ Petal.Width,}
                 \AttributeTok{y =}\NormalTok{ Sepal.Width, }
                 \AttributeTok{color =}\NormalTok{ Species)) }\SpecialCharTok{+}
  \FunctionTok{geom\_point}\NormalTok{()}

\FunctionTok{ggplot}\NormalTok{(iris, }\FunctionTok{aes}\NormalTok{(}\AttributeTok{x =}\NormalTok{ Sepal.Width,}
                 \AttributeTok{y =}\NormalTok{ Petal.Width, }
                 \AttributeTok{color =}\NormalTok{ Species)) }\SpecialCharTok{+}
  \FunctionTok{geom\_point}\NormalTok{()}
\end{Highlighting}
\end{Shaded}

\begin{figure}[h]

{\centering \includegraphics{skeleton_files/figure-latex/exemplo_figura2-1} \includegraphics{skeleton_files/figure-latex/exemplo_figura2-2} 

}

\caption{Dois gráficos na mesma figura utilizando o argumento \tt{fig.show='hold'}.}\label{fig:exemplo_figura2}
\end{figure}

\section{Tabelas}

A função \texttt{kable()} do \texttt{knitr} pode ser utilizada para
gerar tabelas compatíveis com LaTeX a partir de objetos do tipo
\texttt{data.frame} do R. Veja o exemplo abaixo:

\begin{Shaded}
\begin{Highlighting}[]
\NormalTok{knitr}\SpecialCharTok{::}\FunctionTok{kable}\NormalTok{(}\FunctionTok{head}\NormalTok{(iris), }\StringTok{"latex"}\NormalTok{, }
             \AttributeTok{caption =} \StringTok{"Primeiras linhas do dataset \textasciigrave{}iris\textquotesingle{}."}\NormalTok{,}
             \AttributeTok{booktabs =} \ConstantTok{TRUE}\NormalTok{, }\AttributeTok{digits =} \DecValTok{2}\NormalTok{, }\AttributeTok{position =} \StringTok{"h!"}\NormalTok{)}
\end{Highlighting}
\end{Shaded}

\begin{table}[h!]

\caption{\label{tab:exemplo_tabela}Primeiras linhas do dataset `iris'.}
\centering
\begin{tabular}[t]{rrrrl}
\toprule
Sepal.Length & Sepal.Width & Petal.Length & Petal.Width & Species\\
\midrule
5.1 & 3.5 & 1.4 & 0.2 & setosa\\
4.9 & 3.0 & 1.4 & 0.2 & setosa\\
4.7 & 3.2 & 1.3 & 0.2 & setosa\\
4.6 & 3.1 & 1.5 & 0.2 & setosa\\
5.0 & 3.6 & 1.4 & 0.2 & setosa\\
\addlinespace
5.4 & 3.9 & 1.7 & 0.4 & setosa\\
\bottomrule
\end{tabular}
\end{table}

Uma alternativa para gerar tabelas mais customizáveis é o pacote
\texttt{xtable}. Uma galeria de exemplos utilizando o \texttt{xtable}
está disponível em
\url{https://cran.r-project.org/web/packages/xtable/vignettes/xtableGallery.pdf}

Nota: É necessário acrescentar o parâmetro \texttt{results="asis"} ao
chunk para que a tabela seja corretamente interpretada com a sintaxe de
LaTeX com o \texttt{xtable}.

\section{Referenciando objetos gerados com o R}

Para referenciar um objeto incluído no texto a partir de um chunk em R,
basta utilizar o \texttt{label} dado ao chunk. Utilize
\texttt{\textbackslash{}autoref\{tipo:nome\_do\_chunk\}} para inserir o
tipo de referência automaticamente ou
\texttt{\textbackslash{}ref\{tipo:nome\_do\_chunk\}} para apenas o
número. \texttt{tipo} representa o tipo de objeto resultante daquele
chunk como \texttt{fig} para figuras e \texttt{tab} para tabelas.

\subsection{Exemplos}

\begin{itemize}
\tightlist
\item
  \texttt{\textbackslash{}autoref\{fig:exemplo\_figura\}\}} resulta em
  \autoref{fig:exemplo_figura}.
\item
  \texttt{ref\{fig:exemplo\_figura\}} resulta em
  \ref{fig:exemplo_figura}.
\item
  \texttt{\textbackslash{}autoref\{tab:exemplo\_tabela\}} resulta em
  \autoref{tab:exemplo_tabela}.
\item
  \texttt{\textbackslash{}ref\{tab:exemplo\_tabela\}} resulta em
  \ref{tab:exemplo_tabela}.
\end{itemize}

\chapter{Comandos gerais em LaTeX}

Para informações sobre como utilizar a classe
\texttt{imecc-unicamp.cls}, recomendo ver o template em LaTeX
desenvolvido pelo Fábio Rodrigues Silva, no link
\url{https://www.overleaf.com/latex/templates/modelo-tese-imecc-unicamp/hrryjftrmzhz}

Uma vez que toda a sintaxe de LaTeX é interpretável no formato
\texttt{Rmarkdown}, esse modelo é uma versão estendida do modelo em
LaTeX com a possibilidade de acrescentar chunks em R.

\section{Citações}

O modelo utiliza o \texttt{abntex2cite} para processar as citações. Você
pode utilizar:

\begin{itemize}
\tightlist
\item
  \texttt{\textbackslash{}citeonline\{breiman2001statistical\}} para
  referências no meio do texto, resultando em
  \citeonline{breiman2001statistical}.
\item
  \texttt{\textbackslash{}cite\{breiman2001statistical\}} para
  referências entre parênteses \cite{breiman2001statistical}.
\end{itemize}

\phantompart
\chapter{Considerações Finais}

Para tirar qualquer dúvida sobre esse documento, você pode visitar o
repositório \url{https://github.com/Freguglia/imecc.tese} e abrir uma
Issue com a sua pergunta.

\postextual

\bibliography{bibliografia.bib}

\phantompart
\cleardoublepage
\phantomsection
\addcontentsline{toc}{chapter}{\glossaryname}
\printglossaries

\begin{apendicesenv}
\partapendices

\chapter{Primeiro apendice}

\lipsum[50]

\end{apendicesenv}

\begin{anexosenv}
\partanexos

\chapter{Nome do anexo}

\lipsum[50]

\end{anexosenv}

\phantompart
\printindex
\end{document}
